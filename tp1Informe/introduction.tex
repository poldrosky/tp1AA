\section{Introducción}

El objetivo del presente informe es presentar los resultados del análisis del
comportamiento del algoritmo de aprendizaje de árboles de decisión J48 en función del
Confidence Factor (CF) y evaluar aspectos como el sobreajuste y su robustez ante
variaciones en el conjunto de datos. La variaciones en el conjunto de datos que fueron 
tenidas en cuenta fueron: datos faltantes, la tolerancia al ruido y la
discretización de atributos numéricos.

El conjunto de datos utilizado es de las pruebas de estado Saber Pro 2012-1 en Colombia,
uno de los objetivos del examen de estado de calidad de la educación superior Saber
Pro, según el decreto 3963 del 14 de octubre de 2009, Ministerio de Educación
Nacional \cite{men2009}, es comprobar el grado de desarrollo de competencias de los
estudiantes próximos a culminar los programas académicos de pregrado que ofrecen
las instituciones de educación superior. El examen está compuesto por pruebas que
evalúan competencias genéricas y específicas. De acuerdo a los lineamientos Saber
Pro del Instituto colombiano para el Fomento de la Educación Superior \cite{icfes2011},
todos los estudiantes deben presentar los módulos de competencias genéricas sin
importar el programa de formación que cursen, que incluye competencias de
razonamiento cuantitativo, lectura crítica, escritura e inglés.

En la competencia de razonamiento cuantitativo se evalúan los desempeños
relacionados con uso de lenguaje cuantitativo y solución de problemas \cite{icfes2012a}.
En la competencia de lectura crítica se evalúan los desempeños asociados a lectura,
pensamiento crítico y entendimiento interpersonal \cite{icfes2012a}. En escritura se
evalúa la competencia para comunicar ideas por escrito referidas a un tema dado
\cite{icfes2011} \cite{icfes2012a}. En inglés se evalúa la competencia del estudiante para
comunicarse efectivamente en inglés.

El conjunto de datos pertenece a los datos de las pruebas Saber Pro 2012-1, la cual cuenta con
94 variables y 97.068 registros, a este conjunto de datos se le realizó un tratamiento de transformación
de variables para reducir la dimesión, con esto se obtuvo un nuevo conjunto de datos con 31 variables y
96.775 registros. 

Las variables que se usaron para el análisis representan, información personal del estudiante como lo muestra
la tabla~\ref{table:estu}, información familiar del estudiante  como lo muestra la tabla~\ref{table:fami},
información de la institución que cursa el estudiante como lo muestra la tabla~\ref{table:inst} y la información socio-económica
del estudiante como lo muestra la tabla~\ref{table:socioeconomica}.



\begin{table*}
\begin{center}
\caption{Información personal estudiante}
\label{table:estu}
\rowcolors{1}{}{lightgray}
\scalebox{0.8}{%
\begin{tabular}{|>{\centering\arraybackslash}m{2cm}|>{\arraybackslash}m{4cm}|>{\arraybackslash}m{2cm}|>{\arraybackslash}m{3cm}|>{\arraybackslash}m{4cm}| }
\hline
  \rowcolor{blue!55} 
   \multicolumn{1}{|c|}{Atributo/Clase} & \multicolumn{1}{c|}{Nombre} & \multicolumn{1}{c|}{Tipo} & 
   \multicolumn{1}{c|}{Descripción} & \multicolumn{1}{c|}{Estadística} \\ \hline
    Clase & mod\_razona\_cuantitativo & Cualitativa Nominal & Nivel asignado al modulo de Razonamiento Cuantitativo. & mode = BAJO LA MEDIA (48757),
    least = SOBRE LA MEDIA  (48018) \\ \hline
    Atributo & estu\_genero & Cualitativa Nominal & Género alumno. & mode = F – Femenino(40084),least= F – Masculino(56691) \\ \hline
    Atributo & estu\_edad & Cuantitativa & Edad alumno al momento de tomar la prueba. & Min=9.00, 1st Qu=22, Median=24,    Mean=26.03, 3rd Qu=28, Max=74. \\ \hline
    Atributo & estu\_estado\_civil & Cualitativa Nominal & Estado civil alumno. & mode = Soltero(a)(77732), least = Viudo(a)(163) \\ \hline
    Atributo & estu\_hogar\_actual & Cualitativa Nominal & Su hogar actual. & mode = Es el habitual-permanente(79298),
    least = Es temporal por razones de estudio u otra razón(17477) \\ \hline
    Atributo & estu\_sn\_cabeza\_fmlia & Cualitativa Nominal & Es cabeza de familia. & mode = No(80380), least = Si(16395) \\ \hline
    Atributo & estu\_grupo\_referencia & Cualitativa Nominal & Nombre del grupo de referencia al que pertenece el programa
    académico del evaluado. & mode = CIENCIAS ECONOMICAS Y ADMINISTRATIVAS(26557),
    least = ARTES - DISEÑO - COMUNICACION(30) \\ \hline
    Atributo & estu\_pje\_creditos & Cualitativa ordinal & Porcentaje de créditos cursados y aprobados. & mode = MAS DE 90\%(46506), least = MENOS DEL 75\%(2883)  \\ \hline
    Atributo & estu\_titulo\_bto & Cualitativa Nominal & Título de bachiller obtenido. & mode = Académico(73955), least = Técnico(4267) \\ \hline
    Atributo & estu\_financiacion\_matricula & Cualitativa Nominal & Fuente de los recursos con que canceló la Matrícula. & mode = PADRES(38622), least =PROPIO, BECA O SUBSIDIO(232) \\ \hline
    Atributo & estu\_estrato & Cualitativa ordinal & Estrato socioeconómico de la vivienda donde reside actualmente
    su hogar habitual o permanente según el recibo del servicio de energía
    Eléctrica? & mode = Estrato3(36274), least=
    Vive en una zona rural donde no hay estratificación socioeconómica(112) \\ \hline
    Atributo & estu\_trabaja & Cualitativa Nominal & Si el alumno usted actualmente? & mode NO(42914), least = SI, POR SER PRACTICA OBLIGATORIA DEL PROGRAMA(7300) \\ \hline
    Atributo & estu\_metodo\_prgm & Cualitativa Nominal & Metodología del programa académico que pertenece el evaluado. & mode = PRESENCIAL(84059), least = SEMIPRESENCIAL(3) \\ \hline
    Atributo & estu\_area\_conoc & Cualitativa Nominal & Nombre del área de conocimiento a la que pertenece el programa académico del evaluado. & mode = ECONOMIA, ADMINISTRACION, CONTADURIA Y AFINES(27034),
    least = AGRONOMIA VETERINARIA Y AFINES(1523) \\ \hline
    Atributo & num\_estu\_zona & Cualitativa ordinal & Nivel estudiantes por zona  & mode = Media(56900), least=Baja(6408) \\ \hline
  \end{tabular}
}
\end{center}
\end{table*}

\begin{table*}
\begin{center}
\caption{Información familiar estudiante}
\label{table:fami}
\rowcolors{1}{}{lightgray}
\scalebox{0.8}{%
\begin{tabular}{|>{\centering\arraybackslash}m{2cm}|>{\arraybackslash}m{4cm}|>{\arraybackslash}m{2cm}|>{\arraybackslash}m{3cm}|>{\arraybackslash}m{4cm}| }
\hline
  \rowcolor{blue!55} 
   \multicolumn{1}{|c|}{Atributo} & \multicolumn{1}{c|}{Nombre} & \multicolumn{1}{c|}{Tipo} & 
   \multicolumn{1}{c|}{Descripción} & \multicolumn{1}{c|}{Estadística} \\ \hline
    Atributo & fami\_num\_pers\_cargo & Cuantitativa & Tiene personas a cargo (cuando es cabeza de familia). & mode = No(68472), least = Si(28303) \\ \hline
    Atributo & fami\_nivel\_educa\_padres & Cualitativa Nominal & Nivel educativo de los padres. & mode = SECUNDARIA (BACHILLERATO) COMPLETA(19899),least = NINGUNO(661) \\ \hline
    Atributo & fami\_ocup\_madre & Cualitativa Nominal & Cuál es actualmente la ocupación de su madre? (o última si Falleció?). & mode = Hogar r(41120), least = Empleado-con cargo-como-director(a)(1487) \\ \hline
    Atributo & fami\_ocup\_padre & Cualitativa Nominal & Cuál es actualmente la ocupación de su padre? (o última si Falleció?) & mode = trabajador por cuenta propia(23955), Least = Hogar(1943) \\ \hline
    Atributo & fami\_nivel\_sisben & Cualitativa ordinal & Su familia está clasificada en el nivel 1, 2 ó 3 del SISBEN? & mode = No está clasificada por el SISBEN(54353), least = Está clasificada en otro nivel(804)\\ \hline
    Atributo & fami\_ing\_fmliar\_mensual & Cualitativa ordinal & Cuál es el total de ingresos mensuales de su hogar habitual o permanente (por trabajo u otros conceptos) en salarios mínimos:SM-? & mode = DOS SALARIOS(30151), least = SIETE SALARIOS(4033) \\ \hline
    \end{tabular}
}
\end{center}
\end{table*}

\begin{table*}
\begin{center}
\caption{ Información institución  estudiante}
\label{table:inst}
\rowcolors{1}{}{lightgray}
\scalebox{0.8}{%
\begin{tabular}{|>{\centering\arraybackslash}m{2cm}|>{\arraybackslash}m{4cm}|>{\arraybackslash}m{2cm}|>{\arraybackslash}m{3cm}|>{\arraybackslash}m{4cm}| }
\hline
  \rowcolor{blue!55} 
   \multicolumn{1}{|c|}{Atributo} & \multicolumn{1}{c|}{Nombre} & \multicolumn{1}{c|}{Tipo} & 
   \multicolumn{1}{c|}{Descripción} & \multicolumn{1}{c|}{Estadística} \\ \hline
    Atributo & inst\_tipo & Cualitativa Nominal & Tipo institución & mode = PRIVADA(58025), least = REGIMEN ESPECIAL(47) \\ \hline
    Atributo & inst\_caracter\_academico & Cualitativa Nominal & Carácter Académico. & mode = ACADEMICO(73955) ,least = ESCUELA TECNOLOGICA(4267) \\ \hline
    Atributo & inst\_acreditada & Cualitativa Nominal & Institución alumno acreditada? & mode = INSTITUCION NO ACREDITADA(79807), least = INSTITUCION ACREDITADA(16968) \\ \hline
    Atributo & inst\_programa\_zona & Cualitativa Nominal & Zona del programa de estudio del alumno. & mode = BOGOTA(33467) , least = MARINILLA(2) \\ \hline
    Atributo & num\_instituciones\_zona & Cualitativa ordinal & Nivel instituciones por zona  & mode = Alta(49946), least = Baja (19903) \\ \hline
  \end{tabular}
}
\end{center}
\end{table*}

\begin{table*}
\begin{center}
\caption{Información socioeconómica estudiante}
\label{table:socioeconomica}
\rowcolors{1}{}{lightgray}
\scalebox{0.8}{%
\begin{tabular}{|>{\centering\arraybackslash}m{2cm}|>{\arraybackslash}m{4cm}|>{\arraybackslash}m{2cm}|>{\arraybackslash}m{3cm}|>{\arraybackslash}m{4cm}| }
\hline
\rowcolor{blue!55}  
\multicolumn{1}{|c|}{Atributo} & \multicolumn{1}{c|}{Nombre} & \multicolumn{1}{c|}{Tipo} & \multicolumn{1}{c|}{Descripción} & \multicolumn{1}{c|}{Estadística} \\ \hline
Atributo & eco\_condicion\_vivienda & Cualitativa ordinal & Condición económica vivienda. & mode = BUENA(78857), least = REGULAR(2721) \\ \hline
Atributo & eco\_condicion\_hogar & Cualitativa ordinal & Condición económica hogar. & mode = CONDICION VIVIENDA BUENA(53131), least = CONDICION VIVIENDA MALA(9139) \\ \hline
Atributo & eco\_condicion\_transporte & Cualitativa ordinal & Condición económica de transporte. & mode = CONDICION TRANSPORTE PUBLICO(63499), least =CONDICION TRANSPORTE PARTICULAR(33276)\\ \hline
Atributo & eco\_condicion\_tic & Cualitativa ordinal & Condición tecnológica hogar. & mode = CONDICION HOGAR BUENA(85270), least = CONDICION HOGAR MALA(4706) \\ \hline
Atributo & eco\_condicion\_vive & Cualitativa ordinal & Condición hacinamiento vivienda. & mode = SIN HACINAMIENTO(93333), least = HACINAMIENTO CRITICO(445) \\ \hline
\end{tabular}
}
\end{center}
\end{table*}

La elección del conjunto de datos se fundó en las características enunciadas en \cite{mitchell1997machine} con respecto a la clase
de problemas que son apropiados para trabajar con árboles de decisión particularmente el hecho de que cada 
atributo toma un número pequeño de valores posibles. 
